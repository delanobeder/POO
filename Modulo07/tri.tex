\documentclass[11pt]{article}

\usepackage[brazil]{babel} 
\usepackage[latin1]{inputenc} 
\usepackage{alltt}
\usepackage{multicol} 
\usepackage{graphicx}
\usepackage{listings}
\usepackage{amsmath, amssymb}
\usepackage{booktabs}
\usepackage{tcolorbox}

\setlength{\topmargin}{-2.5cm}
\setlength{\textheight}{26.5cm}
\setlength{\oddsidemargin}{-2cm} 
\setlength{\evensidemargin}{-2cm}
\setlength{\textwidth}{7.5in}

\newcommand{\comb}[2]
%%  to be used in math mode 
{\left( \begin{array}{c} #1 \\ #2 \end{array} \right) }
 
\def\ni{\noindent}

\def\idc{\makebox[1.5cm]{}}

\def\espaco{\makebox[5mm]{}}
 
\def\ua{\uparrow}
 
\def\pule{\vspace{0.2cm}} \def\pulao{\vspace{0.5cm}}
 
\newcommand{\primeira}[1] {$#1^{\mbox{\scriptsize\b{a}}}$}
 
\newcommand{\primeiro}[1] {#1$^{\mbox{\scriptsize\b{o}}}$}
 
\begin{document}
\lstset{language=Java}
\lstset{frameround=fttt}
\thispagestyle{empty}
\begin{center}

\hfill {\small Departamento de Computa��o -- Universidade Federal de S�o Carlos} \\
\hspace{1cm}\\

  \textbf{\large 1001507 -- PROGRAMA��O ORIENTADA A OBJETOS}

  \hspace{1cm}\\
  
  \textsc{Professor: Delano Medeiros Beder}

\end{center}

\noindent Essa atividade consiste em implementar (em C++) as classes conforme as
instru��es fornecidas nas quest�es abaixo.  Deve-se utilizar apenas os conceitos
apresentados  em aula  e  seguir as  boas pr�ticas  de  programa��o orientada  a
objetos.

\begin{enumerate}

%%%%%%%%%%%%%%%%%%%%%%%%%%%%%%%%%%%%%%%%%%%%%%%%%%%%%%%%%%%%%%%%%%%%%%%%%%%%%%%%
%%
%%   Q U E S T A O 2=1
%% 
%%%%%%%%%%%%%%%%%%%%%%%%%%%%%%%%%%%%%%%%%%%%%%%%%%%%%%%%%%%%%%%%%%%%%%%%%%%%%%%%

\item  Considere  a classe  {\sf  Triangulo}  que  representa tri�ngulos  e  que
  encontra-se parcialmente  implementada.  Solicita-se nessa quest�o,  a correta
  implementa��o das funcionalidades listadas abaixo:

\begin{enumerate}

\item Inclua  na classe um  construtor �nico  capaz de inicializar  os atributos
  privados.

\item Implemente o m�todo {\sf double getPerimetro()} que retorna o per�metro do
  tri�ngulo (soma dos 3 lados)

\item  Implemente  o  m�todo  {\sf  double getArea()}  que  retorna  a  �rea  do
  tri�ngulo. Considerando um tri�ngulo  de lados a, b, e c,  a �rea do tri�ngulo
  $A$ � calculada por:

\[ A = \sqrt{p \cdot (p-a) \cdot (p-b) \cdot (p-c)} \]

Sendo que $p$ � o semiper�metro (metade do per�metro) do tri�ngulo:

\item  Implemente   (Sobrecarregue)  os  operadores  relacionais   [crit�rio  de
  compara��o -- �rea do tri�ngulo]

\item  Implemente (Sobrecarregue)  o operador  $<<$ na  classe {\sf  Triangulo}.
  Esse operador deve possibilitar a impress�o das informa��es de um tri�ngulo.
  
\end{enumerate}

\begin{tcolorbox}
\begin{quote}
\begin{scriptsize}
\begin{alltt}
class Triangulo \{
private:
   double a, b, c; // lados do triangulo
public:
    // Construtor �nico [a ser implementado] (a)
    // M�todos getters/setters - n�o ser�o implementados
    double getPerimetro() \{ A ser implmentado (b) \}
    virtual double getArea() \{ A ser implementado (c) \}
    bool operator>(const Triangulo& obj) const \{ A ser implementado (d) \}
    bool operator>=(const Triangulo& obj) const \{ A ser implementado (d) \}
    bool operator<(const Triangulo& obj) const \{ A ser implementado (d) \}
    bool operator<=(const Triangulo& obj) const \{ A ser implementado (d) \}
    bool operator==(const Triangulo& obj) const \{ A ser implementado (d) \}
    bool operator!=(const Triangulo& obj) const \{ A ser implementado (d) \}
    friend ostream& operator<< (ostream& os, const Triangulo& triangulo) \{ A ser implementado (e) \}
\};    
\end{alltt}
\end{scriptsize}
\end{quote}
\end{tcolorbox}

\end{enumerate}

\end{document}
